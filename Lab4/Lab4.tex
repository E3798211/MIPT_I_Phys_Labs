\documentclass[14pt]{article}

\usepackage[utf8x]{inputenc}
\usepackage[russian]{babel}
\usepackage{graphicx}
\graphicspath{{images/}}
\DeclareGraphicsExtensions{.pdf,.png,.jpg}

\usepackage{amsmath}
\usepackage{pgfplots}

\usepackage{geometry} % Меняем поля страницы
\geometry{left=2cm}% левое поле
\geometry{right=1.5cm}% правое поле
\geometry{top=2cm}% верхнее поле
\geometry{bottom=2cm}% нижнее поле

\renewcommand{\theenumi}{\arabic{enumi}}% Меняем везде перечисления на цифра.цифра
\renewcommand{\labelenumi}{\arabic{enumi}}% Меняем везде перечисления на цифра.цифра
\renewcommand{\theenumii}{.\arabic{enumii}}% Меняем везде перечисления на цифра.цифра
\renewcommand{\labelenumii}{\arabic{enumi}.\arabic{enumii}.}% Меняем везде перечисления на цифра.цифра
\renewcommand{\theenumiii}{.\arabic{enumiii}}% Меняем везде перечисления на цифра.цифра
\renewcommand{\labelenumiii}{\arabic{enumi}.\arabic{enumii}.\arabic{enumiii}.}% Меняем везде перечисления на цифра.цифра

\begin{document}
\begin{titlepage}
	\begin{center}
		\fontsize{18pt}{20pt}\selectfont
		\textbf{Работа 1.4.5.}	
	
		\vspace{5cm}
		\fontsize{24pt}{25pt}\selectfont
		Изучение колебаний струны
	\end{center}
	\begin{flushright}
		\fontsize{18pt}{20pt}\selectfont
		\vspace{14cm}
		\hspace{-3cm}
		\textit{Корнеев Е.С.}
	\end{flushright}		
\end{titlepage}

\begin{center}
	\fontsize{16pt}{18pt}\selectfont	
	Изучение колебаний струны
\end{center}

\fontsize{14pt}{16pt}\selectfont
\vspace{1cm}
\textbf{Цель работы:} исследование зависимости частоты колебаний струны от величины натяжения, а также условий установления стоячей волны, получающейся в результате сложения волн, идущих в противоположных направлениях.

\vspace{0.5cm}
\textbf{В работе используются:} рейка со струной, звуковой генератор, постоянный магнит, разновесы.

\vspace{1cm}
%Основное свойство струны --- гибкость --- является следствием ее большой длины по сравнению с поперечными размерами.
Струной в акустике называют однородную тонкую гибкую упругую нить. Примерами могут служить сильно натянутый шнур или трос, струны гитары, скрипки и других музыкальных инструментов. В данной работе изучаются поперечные колебания стальной гитарной струны, натянутой горизонтально и закрепленной между двумя неподвижными зажимами.

Основное свойство струны --- гибкость --- обусловлено тем, что её поперечные размеры малы по сравнению с длиной. Это означает, что напряжение в струне может быть направлено только вдоль неё, и позволяет не учитывать изгибные напряжения, которые могли бы возникать при поперечных деформациях (то есть, при изгибе струны).

В натянутой струне возникает \textsl{поперечная упругость}, т.е. способность сопротивляться всякому изменению формы, происходящему без изменения обьема. При вертикальном смещении произвольного элемента струны, возникают силы, действующие на соседние элементы, и в резульатте вся струна приходит в движение в вертикальной плоскости, т.е. возбуждение «бежит» по струне. Передача возбуждения представляет собой поперечные бегущие волны, распространяющиеся с некоторой скоростью в обе стороны от места возбуждения. В ненатянутом состоянии струна не обладает свойством поперечной упругости и поперечные волны на ней невозможны.


%%%%%%%%%%%%%%%%%%%%%%%%%%%%%%%%%%%
%
%	РИСУНОК СО СТРУНОЙ, НОМЕР 1
%
%%%%%%%%%%%%%%%%%%%%%%%%%%%%%%%%%%%



Рассмотрим гибкую однородную струну, в которой создано натяжение $T$, и получим дифференциальное уравнение, описывающее её малые поперечные свободные колебания. Отметим, что если струна расположена горизонтально в поле тяжести, величина $T$ должна быть достаточна для того, чтобы в
состоянии равновесия струна не провисала, т. е. сила натяжения должна существенно превышать вес струны.

Направим ось $x$ вдоль струны в положении равновесия. Форму струны будем описывать функцией $y(x, t)$, определяющей её вертикальное смещение в
точке $x$ в момент времени $t$ (рис. 1). Угол наклона касательной к струне в точке $x$ относительно горизонтального направления обозначим как 
$\alpha$. В любой момент этот угол совпадает с углом наклона касательной к графику функции $у(х)$, то есть 
$tg~a = \frac{\partial y}{\partial y}$.

Рассмотрим элементарный участок струны, находящийся в точке $x$, имеющий длину $\delta x$ и массу $\delta m = \rho \delta x$, где $\rho$ --- погонная плотность струны (масса на единицу длины). При отклонении от равновесия на выделенный элемент действуют силы натяжения $T_1$ и $T_2$, направленные по касательной к струне. Их вертикальная составляющая будет стремиться вернуть рассматриваемый участок струны к положению равновесия, придавая элементу некоторое вертикальное ускорение $\frac{\partial^2y}{\partial t^2}$. Заметим, что угол $\alpha$ зависит от координатs $x$ вдоль струны и различен в точках приложения сил $T_1$ и $T_2$. Таким образом, второй закон Ньютона для вертикального движения элемента струны запишется в следующем виде:

\begin{center}
\begin{equation}
\delta m \frac{\partial^2y}{\partial t^2} = -T_1\sin \alpha_1 + T_2\sin \alpha_2
\end{equation}
\end{center}


Основываясь на предположении, что отклонения струны от положения равновесия малы, можно сделать ряд упрощений:

1. Длина участка струны в изогнутом состоянии практически равна длине участка в положении равновесия, поэтому добавочным напряжением вследствие удлинения струны можно пренебречь. Следовательно, силы $T_1$ и $T_2$ по модулю равны силе натяжения струны: $T_1 \approx T_2 \approx T$.

2. Углы наклона $\alpha$ малы, поэтому $tg~\alpha \approx \sin\alpha \approx \alpha$ и, следовательно, можно положить $\alpha \approx \frac{\delta y}{\delta x}$.

Разделим обе части уравнения (1) на $\delta x$ и перейдем к дифференциалам:

\begin{center}
\begin{equation}
\rho \frac{\partial^2 y}{\partial t^2} = \frac{T_2\sin\alpha_2 - T_1\sin\alpha_1}{\delta x} = T\frac{\partial a}{\partial x}
\end{equation}
\end{center}

Подставим $\alpha = \frac{\partial y}{\partial x}$ и введем обозначение $u = \sqrt{\frac{T}{\rho}}$, что, как мы увидим далее, является скоростью распространения волн на струне, находим уравнение свободных малых поперечных колебаний струны:


\begin{center}
\begin{equation}
\boxed{\frac{\partial^2 y}{\partial t^2} = u^2 \frac{\partial^2 y}{\partial x^2}}
\end{equation}
\end{center}

Уравнение (3) называется \textsl{волновым уравнением}. Кроме волн на струне, оно может описывать волновые процессы в самых разных системах, в том числе волны в сплошных средах (звук), электромагнитные волныи т.д. 

\vspace{1cm}
\textbf{Бегущие волны}

Как показывается в математических курсах, общее решение дифференциального уравнения в частных производных (3) представимо в виде суммы двух волн произвольной формы, бегущих в противоположные стороны со скоростями $u$:

\begin{center}
\begin{equation}
y(x, t) = y_1(x - ut) + y_2(x + ut)
\end{equation}
\end{center}

\noindent где $u$ --- скорость распространения волны, $y_1$ и $y_2$ --- произвольные функции, вид которых определяется из начальных и граничных условий. При этом особый случай представляет случай гармонических волн:

\begin{center}
\begin{equation}
y(x, t) = a\cos[k(x - ut)] + b\cos[k(x + ut)] = a\cos(\omega t - kx) + b\cos(\omega t + kx)
\end{equation}
\end{center}

\noindent Выражение (5) представляет собой суперпозицию двух гармонических волн, бегущих навстречу друг другу со скоростью 

\begin{center}
\begin{equation}
u = \frac{\omega}{k} = \nu\lambda
\end{equation}
\end{center}

При этом длина волны $\lambda = \frac{2\pi}{k}$, частота $\nu = \frac{\omega}{2\pi}$. Величина $k = \frac{2\pi}{\lambda}$ называется \textsl{волновым числом} или \textsl{пространственной частотой волны}.

Заметим, что формула (2) означает, что скорость распространения волны зависит только от силы натяжения струны $T$ и погонной плотности $\rho$.

\vspace{1cm}
\textbf{Собственные колебания струны. Стоячие волны}

Найдем вид свободных колебаний струны с закрепленными концами. Пусть струна закреплена в точках $x = 0$ и $x = L$. Концы струны не колеблются, поэтому $y(0, t) = 0$ и $y(L, t) = 0$ для любых $t$. Используя (5), находим

$$y(0, t) = a\cos\omega t + b\cos\omega t = 0$$

\noindent откуда следует, что $a = -b$. Тогда после тригонометрических преобразований выражение (5) примет вид

\begin{center}
\begin{equation}
y(x, t) = 2a\sin kx \cdot \sin\omega t
\end{equation}
\end{center}

Колебания струны, описываемые функцией (7), называются \textsl{стоячими волнами}. Видно, что стоячая волна может быть получена как сумма (интерференция) двух гармонических бегущих волн, имеющих равную амплитуду и движущихся навстречу друг другу. Как видно из (7), точки струны, в которых $\sin kx = 0$, в любой момент времени неподвижны. Такие точки называются узлам. Остальные точки совершают в вертикальной плоскости гармонические колебания с частотой

$$\nu = \frac{\omega}{2\pi} = \frac{u}{\lambda}$$

Амплитуда колебаний распределена вдоль струны по гармоническому закону: $y_0(x) = 2a\sin kx$. В точках, где $\sin kx = 1$, амплитуда колебаний максимальна — они называются \textsl{пучностями}. Между двумя соседними узлами все участки струны колеблются в фазе (их скорости имеют одинаковое направление), а при переходе через узел фаза колебаний меняется на $\pi$ вследствие изменения знака $\sin kx$.

Используя второе граничное условие $y(L, t) = 0$ (точки крепления струны должны быть узлами стоячей волны), найдём условие образования стоячих
волн на струне: $y(x, t) = 2a\sin kL \sin \omega t = 0$, откуда 

$$\sin kL = 0 \Rightarrow kL = \frac{\pi}{2}n, n \in N$$

Таким образом, стоячие волны на струне с закреплёнными концами могут образовываться только если на длине струны укладывается целое число полуволн:

\begin{center}
\begin{equation}
L = \frac{\lambda}{2}n
\end{equation}
\end{center}

Поскольку длина волны однозначно связана с её частотой, струна может колебаться только с определёнными частотами:

\begin{center}
\begin{equation}
\nu_n = \frac{u}{\lambda_n} = \frac{n}{2L}\sqrt{\frac{T}{\rho}}, n \in N
\end{equation}
\end{center}

Набор (спектр) разрешённых частот $\nu_n$ называют \textsl{собственными частотами} колебаний струны. Режим колебаний, соответствующий каждой из частот $\nu_n$, называется собственной (или нормальной) \textsl{модой} колебаний (от англ. mode --- режим). Произвольное колебание струны может быть представлено в виде суперпозиции её собственных колебаний. Наименьшая частота $\nu_n$ называется также \textsl{основным тоном} (или первой гармоникой), а остальные ($\nu_2 = 2\nu_1$, $\nu_3 = 3\nu_1$, ...) --- \textsl{обертонами} (высшими гармониками). Термин «гармоника» иногда употребляется в обобщенном смысле — как элементарная составляющая сложного гармонического колебания.

На рис. 2 показана картина стоячих волн для $n = 1, 2, 3$. Заметим, что число $n$ определяет число пучностей (но не узлов!) колеблющейся струны. Таким образом, спектр собственных частот струны определён её погонной плотностью $\rho$, силой натяжения $T$ и длиной струны $L$ (отдельно отметим, что собственные частоты не зависят от модуля Юнга материала струны).

%%%%%%%%%%%%%%%%%%%%%%%%%%%%%%%%%%%%
%
%	КАРТИНКА 2
%
%%%%%%%%%%%%%%%%%%%%%%%%%%%%%%%%%%%%

\vspace{1cm}
\textbf{Возбуждение колебаний струны}

При колебаниях реальной струны всегда имеет место потеря энергии (часть теряется вследствие трения о воздух; другая часть уходит через неидеально закрепленные концы струны и т.д.). Поддержание незатухающих колебаний в струне может осуществляться точечным источником, в качестве которого в данной работе используется электромагнитный вибратор. При этом возникает необходимость переноса энергии от источника по всей струне.

Рассмотрим вопрос о передаче энергии по струне. В стоячей волне поток энергии вдоль струны отсутствует — колебательная энергия, заключенная
в отрезке струны между двумя соседними узлами, не транспортируется в другие части струны. В каждом таком отрезке происходит периодическое (два-
жды за период) превращение кинетической энергии в потенциальную и обратно. Передача энергии между различными участками струны возможна только благодаря бегущим волнам, которые, однако, в рассмотренной выше идеальной модели струны не возникают. Парадокс снимается, если учесть, что из-за потерь энергии при отражении волны от концов не происходит полной компенсации падающей и отраженной волны, поэтому к стоячей волне на струне добавляется малая бегущая компонента — именно она служит «разносчиком» энергии по всей системе. 

Для эффективной раскачки колебаний используется явление резонанса --- вынуждающая частота $\nu$ должна совпадать с одной из собственных частот струны $\nu_n$ (9). Когда потери энергии в точности компенсируются энергией, поступающей от вибратора, колебания струны становятся стационарными и на ней можно наблюдать стоячие волны. Если потери энергии за период малы по сравнению с запасом колебательной энергии в струне, то искажение стоячих волн бегущей волной не существенно --- наложение бегущей волны малой амплитуды на стоячую визуально приводит к незначительному «размытию» узлов.

Для достижения максимального эффекта от вибратора, его следует располагать вблизи узловой точки. Это можно показать из следующих элементарных соображений. Пусть вибратор, размещённый в точке $x_0$, способен раскачать соответствующий элемент струны до амплитуды $A$. Если частота вибратора близка к резонансной (т.е. собственной), то как следует из (7), амплитуда колебаний струны в пучности будет равна 
$2a = \frac{A}{\sin kx_0}$. Таким образом, максимальная раскачка струны достигается, если значение $\sin kx_0$ устремить к нулю, что и соответствует положениям узлов (из идеализированной модели струны следует, что при размещении вибратора в узле амплитуда колебаний устремится к бесконечности, однако в реальности она ограничивается силами трения и нелинейными эффектами). Заметим также, что при наблюдении стоячих волн важно, чтобы колебания происходили в одной (вертикальной) плоскости, т.е. были поляризованы. Кроме того, важно, чтобы колебания струны происходили с малой амплитудой, поскольку при сильном возбуждении нарушаются условия применимости волнового уравнения (3), и в опыте наблюдаются искажения, связанные с нелинейными эффектами. 









\end{document}