\documentclass[14pt]{article}

\usepackage[utf8x]{inputenc}
\usepackage[russian]{babel}
\usepackage{graphicx}
\graphicspath{{images/}}
\DeclareGraphicsExtensions{.pdf,.png,.jpg}

\usepackage{amsmath}
\usepackage{multirow}
\usepackage{pgfplots}

\usepackage{geometry} % Меняем поля страницы
\geometry{left=2cm}% левое поле
\geometry{right=1.5cm}% правое поле
\geometry{top=2cm}% верхнее поле
\geometry{bottom=2cm}% нижнее поле

\renewcommand{\theenumi}{\arabic{enumi}}% Меняем везде перечисления на цифра.цифра
\renewcommand{\labelenumi}{\arabic{enumi}}% Меняем везде перечисления на цифра.цифра
\renewcommand{\theenumii}{.\arabic{enumii}}% Меняем везде перечисления на цифра.цифра
\renewcommand{\labelenumii}{\arabic{enumi}.\arabic{enumii}.}% Меняем везде перечисления на цифра.цифра
\renewcommand{\theenumiii}{.\arabic{enumiii}}% Меняем везде перечисления на цифра.цифра
\renewcommand{\labelenumiii}{\arabic{enumi}.\arabic{enumii}.\arabic{enumiii}.}% Меняем везде перечисления на цифра.цифра

\begin{document}
\begin{titlepage}
	\begin{center}
		\fontsize{18pt}{20pt}\selectfont
		\textbf{Работа 1.4.4.}	
	
		\vspace{5cm}
		\fontsize{24pt}{25pt}\selectfont
		Исследование свободных колебаний связанных маятников
	\end{center}
	\begin{flushright}
		\fontsize{18pt}{20pt}\selectfont
		\vspace{14cm}
		\hspace{-3cm}
		\textit{Корнеев Е.С.}
	\end{flushright}		
\end{titlepage}

\begin{center}
	\fontsize{16pt}{18pt}\selectfont	
	Исследование свободных колебаний связанных маятников
\end{center}

\fontsize{14pt}{16pt}\selectfont
\vspace{1cm}
\textbf{Цель работы:} изучение колебательной системы с двумя степенями свободы.

\vspace{0.5cm}
\textbf{В работе используются:} устновка с двумя одинаковыми математическими маятниками, бифилярно подвещенными на натянутую горизонтально струну, секундомер, измерительная линейка.

\vspace{1cm}
\textbf{Свободные колебания связанных маятников.} Рассмотрим простейшую модель с двумя степенями свободы --- два одинаковых маятника, связанных пружиной и совершающих колебания в плоскости рисунка. Маятники представляют собой невесомые спицы с насаженными на них маленькими тяжелыми шариками.

Обозначения указаны на рисунке. Если углы отклонения маятника от положения равновесия достаточно малы 
($\sin\varphi \approx \varphi,~\cos\varphi \approx 1 - \varphi/2$), то со стороны пружины на первый маятник действует момент силы, равный

$$M_{21} = ka^2(\varphi_2 - \varphi_1)$$

\noindent Аналогично второй маятник будет испытывать вращающий момент противоположного знака:

$$M_{12} = -ka^2(\varphi_2 - \varphi_1)$$

\noindent Эти моменты описывают связь между маятниками.

Уравнения движения матяником имеют вид

\begin{equation}\label{fluct_eq_1}
ml^2\frac{d^2\varphi_1}{dt^2} = -mgl\varphi_1 + ka^2(\varphi_2 - \varphi_1)
\end{equation}	
\begin{equation}\label{fluct_eq_2}
ml^2\frac{d^2\varphi_2}{dt^2} = -mgl\varphi_2 - ka^2(\varphi_2 - \varphi_1)
\end{equation}

\noindent Сложив эти два уравнения, находим

\begin{equation}\label{fluct_sum}
ml^2\frac{d^2}{dt^2}(\varphi_1 + \varphi_2) = -mgl(\varphi_1 + \varphi_2)
\end{equation}

Вычитание (\ref{fluct_eq_2}) - (\ref{fluct_eq_1}) дает

\begin{equation}\label{fluct_div}
ml^2\frac{d^2}{dt^2}(\varphi_1 - \varphi_2) = -(mgl + 2ka^2)(\varphi_1 - \varphi_2)
\end{equation}


%%%%%%%%%%%%%%%%%%%%%%%%%%%%%%%%%
%
% РИСУНОК 4.6 (стр. 232)
%
%%%%%%%%%%%%%%%%%%%%%%%%%%%%%%%%%

Решения уравнений (\ref{fluct_eq_1}) и (\ref{fluct_eq_2}) имеют вид:

\begin{equation}\label{fluct_eq_solve_1}
\varphi_1 + \varphi_2 = A\cos(\omega^+t + \alpha)
\end{equation}
\begin{equation}\label{fluct_eq_solve_2}
\varphi_1 - \varphi_2 = B\cos(\omega^-t + \beta)
\end{equation}

$$\omega^+ = \sqrt{\frac{g}{l}}, ~ \omega^- = \sqrt{\frac{g}{l} + \frac{2ka^2}{ml^2}}$$

\noindent где $A, B, \alpha, \beta$ - произвольные константы. Сладывая и вычитая (\ref{fluct_eq_solve_1}) и (\ref{fluct_eq_solve_2}), находим

\begin{equation}\label{ph_1}
\varphi_1 = \frac{1}{2}A\cos(\omega^+t + \alpha) + \frac{1}{2}B\cos(\omega^-t + \beta)
\end{equation}
\begin{equation}\label{ph_2}
\varphi_2 = \frac{1}{2}A\cos(\omega^+t + \alpha) - \frac{1}{2}B\cos(\omega^-t + \beta)
\end{equation}

\noindent Для угловых скоростей при этом имеем

\begin{equation}
\dot \varphi_1 = -\frac{1}{2}A\sin(\omega^+t + \alpha) - \frac{1}{2}B\sin(\omega^-t + \beta)
\end{equation}
\begin{equation}
\dot \varphi_2 = -\frac{1}{2}A\sin(\omega^+t + \alpha) + \frac{1}{2}B\sin(\omega^-t + \beta)
\end{equation}

Проанализируем полученные решения.


%%%%%%%%%%%%%%%%%%%%%%%%%%%%%%%%%%%%%
%
%
%
%
%
%
%
%	СТРАНИЦА 233 
%
%
%
%
%
%
%
%%%%%%%%%%%%%%%%%%%%%%%%%%%%%%%%%%%%%










\setcounter{equation}{0}

%
% Here is new part of the book taken from 1.4.4.
%

Измерения проводятся на установке, изображенной на на рис. 1.

Один конец струны прикреплен к вертикальной стойке установки, а другой конец переброшен через неподвижный блок и натянут при помощи груза массы 
$M$. Точки струны $A$ и $B$ неподвижны. В точках $C$ и $D$, которые делят расстояние между $A$ и $B$ на три равные части (каждая длиной $a$), подвешены одинаковые математические маятники массой $m$ и длиной $l$. Каждый маятник подвешен на двух нитях в плоскости струны (бифилярно), чтобы колебания мятников проходили в плоскостях, перпендикулярных струне. Сила натяжения струны намного больше веса матяников ($M \gg m$). Вертикальная составляющая смещения струны никак не сказывается на движении маятников при малых отклонениях. Горизонтальная составляющая смещения струны, хоть она и мала по сравнению со смещениями маятников, осуществляет слабую связь между маятниками.

На рис. 2. показаны смещения точек $C$ и $D$ струны и отклонения маятников в вертикальной (рис. 2а) и горизонтальной (рис. 2б) плоскостях. 

При небольших отклонениях маятников для силы натяжения подвеса маятника $T$ имеем (рис. 2а)

\begin{equation}
mg \approx T
\end{equation}

Для движения матяников в горизонтальном направлении (рис. 2)

\begin{equation}
m\ddot x_1 = -T\sin\varphi_1 \approx -T\frac{x_1 - x_3}{l} \approx -mg\frac{x_1 - x_3}{l}
\end{equation}
\begin{equation}
m\ddot x_2 = -T\sin\varphi_2 \approx -T\frac{x_2 - x_4}{l} \approx -mg\frac{x_2 - x_4}{l}
\end{equation}

Связь между натяжением струны и натяжением подвеса получаем из рис. 2:

\begin{equation}
T\frac{x_1 - x_3}{l} = F\frac{x_3}{a} + F\frac{x_3 - x_4}{a}
\end{equation}
\begin{equation}
T\frac{x_2 - x_4}{l} = F\frac{x_4}{a} + F\frac{x_4 - x_3}{a}
\end{equation}

Введем безразмерный параметр

\begin{equation}
\sigma = \frac{T}{F} \frac{a}{l} = \frac{m}{M}\frac{a}{l}
\end{equation}

\noindent который в нашем случае много меньше единицы (связь слабая). Тогда из (4) и (5) получаем

\begin{equation}
\sigma x_1 = (2 + \sigma)x_3 - x_4,~~\sigma x_2 = (2 + \sigma)x_4 - x_3
\end{equation}

\noindent пренебрегая $\sigma$ по сравнению с 2, получаем

\begin{equation}
x_3 = \sigma\frac{2x_1 + x_2}{3},~~ x_4 = \sigma\frac{x_1 + 2x_2}{3}
\end{equation}

Уравнения движения маятников примут вид

\begin{equation}
\ddot x_1 + \frac{g}{l}(1 - \sigma)x_1 = \sigma\frac{g}{3l}(x_2 - x_1)
\end{equation}
\begin{equation}
\ddot x_2 + \frac{g}{l}(1 - \sigma)x_2 = \sigma\frac{g}{3l}(x_1 - x_2)
\end{equation}








\end{document}